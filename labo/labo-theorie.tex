\documentclass{article}

\usepackage[dutch]{babel}
\usepackage[margin=3cm]{geometry}
\usepackage{graphicx}
\usepackage{float}
\usepackage{caption}
\usepackage{hyperref}
\usepackage{amsmath}
\usepackage{wrapfig}
\usepackage[parfill]{parskip}
\usepackage{listings}
\usepackage{xcolor}

\lstset{
basicstyle=\ttfamily,
frame=single,
breakatwhitespace=false,         % sets if automatic breaks should only happen at whitespace
breaklines=true,                 % sets automatic line breaking
commentstyle=\color{blue},
showstringspaces=false
}

% fonts
\usepackage[T1]{fontenc}
\usepackage{helvet}
\renewcommand{\familydefault}{\sfdefault}
 
\graphicspath{{img/}} 

\newcommand{\bold}[1]{\textbf{#1}}

\begin{document}

\begin{titlepage}
    \author{Tuur Vanhoutte}
    \title{Sensors \& Interfacing: labo}
\end{titlepage}

\pagenumbering{gobble}
\maketitle
\newpage
\tableofcontents
\newpage

\pagenumbering{arabic}

\section{One-wire bus}
\begin{itemize}
    \item Ontworpen om over lange afstanden kleine hoeveelheden data te sturen. 
    \item Kan makkelijk in kleine, goedkope sensoren ingebouwd worden
    \item Voorzien van 3 pootjes (3 draden): 2 voor voeding (+ en -), 1 voor data
    \item 1 master en 1 of meerdere slaves, elke slave heeft een uniek 64bit adres of slave ID
\end{itemize}




\subsection{Schema}

\begin{figure}[H]
    \centering
    \includegraphics[width=0.5\textwidth]{Screenshot_20200619_150049.png}
    \caption{Schema 1-wire bus met meerdere slaves}
\end{figure}

\subsection{Timingdiagram}
Omdat er maar 1 draad beschikbaar is voor communicatie, moet het protocol vertrouwen op erg strikte timing.
Raspbian voorziet een driver die daarvoor zorgt.

\begin{figure}[H]
    \centering
    \includegraphics[width=0.5\textwidth]{timing-1w.png}
    \caption{Timingdiagram}
\end{figure}

\subsection{Werking}
\subsubsection{Open-collector outputs}
Het is niet mogelijk om zomaar een digitale 0 voor te stellen als 0V, en een digitale 1 als 3.3V.
Dat kan zorgen voor kortsluiting. Daarom gebruiken we een open-collector uitgang:

\begin{figure}[H]
    \centering
    \includegraphics[width=0.5\textwidth]{open-collector.png}
    \caption{Interne schakeling van een open-collector uitgang}
\end{figure}

In plaats van slaves toe te laten zelf een spanning op de bus te leggen, brengen we hem standaard op
een hoog signaalniveau d.m.v. een \bold{externe pull-up weerstand}. 
De uitgangen van ICs die op de bus communiceren worden dan voorzien van een \bold{extra transistor}
waarvan de emitter (of source bij een FET) verbonden is met massa. 
De basis (of gate) is verbonden met de uitgang van de logica en de collector (of drain)
met de uitgangspin en dus de bus.

\subsubsection{Data verzenden}
De bus wordt via een externe weerstand verbonden met de voedingsspanning.
Aangesloten apparaten in rust kunnen we beschouwen als een open circuit (de uitgangstransistor geleidt niet).
De bus blijft dus standaard op een hoog signaalniveau en er vloeit (in theorie) geen stroom.

Om data te verzenden, kan een deelnemer de transistor in geleiding brengen en zo de bus verbinden met
massa. Op dit moment zal er een stroom vloeien, door de pull-up weerstand en uitgangstransistor naar massa. 
De spanning op de bus bedraagt op dat moment (bijna) 0V, gegeven door de spanningsdeler tussen de 
externe pull-up weerstand en de restweerstand van de transistor in geleiding.

\begin{figure}[H]
    \centering
    \includegraphics[width=0.5\textwidth]{Screenshot_20200619_152619.png}
    \caption{Meerdere open-collector outputs (hier voorgesteld door schakelaars), op dezelfde bus}
\end{figure}

Wanneer nu een andere deelnemer op hetzelfde moment gaat communiceren, kan ook die enkel een
verbinding van de bus naar massa maken. Elektrisch vormt dit geen enkel probleem, maar de communicatie
van beide deelnemers zal uiteraard nog steeds verstoord zijn.


\subsection{Voordelen en nadalen}
\subsubsection{Voordelen}
\begin{itemize}
    \item \bold{Geen elektrische conflicten}: Meerdere deelnemers kunnen enkel meerdere verbindingen naar massa maken.
    \item \bold{Flexibele spanning}: Omdat enkel de uitgangstransistor elektrisch met de bus verbonden is, is het
    makkelijk om chips te maken die een verschillende spanning op de bus t.o.v. hun voedingsspanning
    kunnen tolereren.
\end{itemize}

\subsubsection{Nadelen}
\begin{itemize}
    \item \bold{Hoog stroomverbruik}: Een laag signaalniveau betekent een constante stroom door de geleider, en ook
    bij hoog signaalniveau zal er nog een kleine lekstroom door de transistors vloeien.
\end{itemize}

\section{Bitoperaties events}
\subsection{Bits \& bytes}
\begin{figure}[H]
    \centering
    \includegraphics[width=0.5\textwidth]{bits.png}
    \caption{1 byte. Least Significant Bit (LSB) = rechts, Most Significant Bit (MSB) = links}
\end{figure}

\subsubsection{Hexadecimale notatie}
\begin{figure}[H]
    \centering
    \includegraphics[width=0.5\textwidth]{hexa.png}
    \caption{Omzetting van binair naar hexadecimaal. 1 nibble = 4 bits}
\end{figure}

\subsection{Python}
\begin{itemize}
    \item bin(x) om waarden om te zetten naar binair
    \item hex(x) om waarden om te zetten naar hexadecimaal
    \item int(x) om waarden om te zetten naar decimaal
\end{itemize}

\subsection{Bitwise Operators}
\begin{itemize}
    \item And: als beide inputs == 1, output = 1, anders 0
    \item Or: als 1 van de inputs == 1, output 1, anders 0
    \item Not: draai input om: 1 wordt 0, 0 wordt 1
    \item Xor: als precies 1 van de inputs == 1, output = 1, anders 0
\end{itemize}

\begin{table}[H]
\begin{tabular}{|l|l|}
\hline
\multicolumn{1}{|c|}{\textbf{Bitwise operator}} & \multicolumn{1}{c|}{\textbf{Symbol in python}} \\ \hline
and                                             & \&                                             \\ \hline
or                                              & |                                              \\ \hline
not                                             & $\sim$                                         \\ \hline
xor                                             & \textasciicircum{}                             \\ \hline
\end{tabular}
\end{table}

\begin{figure}[H]
    \centering
    \includegraphics[width=0.25\textwidth]{bitoperator-examples.png}
    \caption{Voorbeelden bitoperaties}
\end{figure}

\subsubsection{Set bits}
De OR-operatie kan je gebruiken om bits aan te zetten, ongeacht hun huidige waarde:
\begin{lstlisting}[language=Python]
>>> have = 0b11001010 # het register nu 
>>> mask = 0b00000100 # dus maak ik een getal waar ENKEL bit 2 "aan" is
>>> xxxx = 0b11001110 # wat ik wil nu wil uitkomen

# en ik gebruik dat met de |-operator om de bit "aan" te zetten
>>> result = have | mask 
>>> xxxx == result
True 
\end{lstlisting}

Hetzelfde principe kan je gebruiken met de \&-operator om bits op 0 te zetten,
of bits te filteren (stel dat je alleen bit 2 nodig hebt, dan is je masker 0b00000100)

\subsubsection{Toggle bits}
Togglen van een bit kan je doen met de XOR-operatie (\textasciicircum{})

\subsubsection{Bitwise NOT}
Het resultaat van de NOT-operatie op een binair getal is niet het omgekeerde van het getal, 
maar het negatieve getal. Om dat op te lossen moet je het negatieve getal AND-en met 0xFF.

\subsubsection{Shift left/right}
We kunnen bits opschuiven met $<<$ en $>>$. Dit is hetzelfde als vermenigvuldigen met 2 en delen met 2.

\begin{lstlisting}[language=Python]
>>> 0b00010100 >> 4 == 0b00000001
True
>>> 0b11001010 << 4 == 0b110010100000
True
>>> 0b11001010 << 8 == 0b1100101000000000
True
\end{lstlisting}

\subsubsection{In-place operators}
\begin{lstlisting}[language=Python]
>>> x += 2       # x = x + 2
>>> x -= 2       # x = x - 2
>>> x |= 4       # x = x | 0x04 --> bit 2 is nu 1
>>> x &= 253     # x = x & 0xfd --> bit 1 is nu 0
>>> x ^= 0xf0    # x = x ^ 0xf0 --> 4 MSB worden getoggled
>>> x <<= 2      # x = x << 2 --> alle bits 2 plaatsen naar links
>>> x >>= 3      # x = x >> 3 --> alle bits 3 plaatsen naar rechts
\end{lstlisting}

\subsubsection{Volgorde van bewerkingen}
\begin{enumerate}
    \item not $\sim$
    \item shift $<<$, $>>$
    \item and \&
    \item xor \textasciicircum{}
    \item or |
\end{enumerate}

Gebruik haakjes!

\subsection{Binary Coded Decimals (BCD)}
Bij BCD worden 4 bits gebruikt om 1 decimaal cijfer weer te geven

Voor het getal 91 worden dus groepjes van 4 bits gebruikt om elk getal weer te geven:
\begin{lstlisting}
Decimal:       9     1
Binary:     1001  0001
\end{lstlisting}

\bold{Opgelet}: je mag niet zomaar wiskundige operaties toepassen op een BCD code.
De decimale waarde van BCD-getal 91 is 128+32+1 = 161

\subsection{Interrupts \& Events}
\subsubsection{Interrupts}
Een interrupt is een voorziening om de processor op de hoogte te stellen van een bepaalde
gebeurtenis. 
De CPU zal daarop zijn huidige werk onderbreken en een Interrupt Service Routine (ISR),
ook wel Interrupt Handler genoemd, uitvoeren. 

Na afloop gaat hij weer verder met de taak waar hij
voordien aan werkte. Op deze manier hoeft er geen CPU-power verspild te worden met het
wachten op gebeurtenissen.

\subsubsection{Events}
= signaalovergang

\begin{lstlisting}[language=Python]
GPIO.add_event_detect(channel, GPIO.RISING) # add rising edge detection

do_something_else() # thread is bezig met andere dingen

if GPIO.event_detected(channel):
    print('Button released')
    # dit werkt nog altijd
\end{lstlisting}


\subsubsection{Callbacks}
\begin{lstlisting}[language=Python]
def my_callback(channel):
    print('This is a edge event callback function!')
    print('Edge detected on channel %s'%channel)
    print('This is run in a different thread to your main program')

GPIO.add_event_detect(channel, GPIO.RISING, callback=my_callback)
\end{lstlisting}

Je kan meerdere callbacks op eenzelfde pin registeren door gewoon meerdere keren de methode GPIO.add\_event\_callback(\dots) op te roepen.

\bold{Bouncetime}

Bijkomende parameter bij add\_event\_detect(): aantal milliseconden dat dient om de knop te ontdenderen (debounce). 


\section{Seriele communcatie}
Bv: communicatie tussen Arduino en Raspberry Pi. 

\bold{Opgelet:} de Pi werkt op 3.3V en de meeste arduino's op 5V. 
We moeten een levelshifter gebruiken.

\subsection{Serieel protocol}
Zeer eenvoudig om full-duplex te werken:

\begin{itemize}
    \item een draad om data te versturen
    \item een draad om data te ontvangen
    \item gemeenschappelijke massa (GND)
    \item Het is een point-to-point verbinding: 2 deelnemers, dus geen adressering nodig
\end{itemize}

De chip die zo'n seriele interface aanstuurt noemt men een UART (Universal Asynchronous Receiver / Transmitter).
De termen UART, RS-232 en seriele poort worden vaak door elkaar gebruikt. RS-232 of Recommended Standard 232
omschrijft connectoren, elektrische eigenschappen \dots

\subsection{Hardware (Layer 1)}
\begin{itemize}
    \item Oorspronkelijk:
    \begin{itemize}
        \item 0 = 3 tot 15V
        \item 1 = -3 tot -15V
    \end{itemize}
    \item Tegenwoordig:
    \begin{itemize}
        \item Veel devices werken op -5V tot +5V
        \item Hetzelfde protocol maar op TTL-niveau (0-5V)
    \end{itemize}
    \item De bus is standaard hoog, wat door de UART zelf geregeld wordt. We moeten dus geen pull-up voorzien.
\end{itemize}

\begin{figure}[H]
    \centering
    \includegraphics[width=0.4\textwidth]{ttl-vs-rs232.png}
    \caption{TTL (boven) vs RS-232 (onder)}
\end{figure}

\subsubsection{Crossed cable of null-modemkabel}
De TX-pin van de zender moet verbonden worden met de RX-pin van de ontvanger en vice versa. 
In de context van de oude seriele poort noemt men zo'n crossed cable een null-modemkabel.

Seriële verbindingen werden veel gebruikt voor het verbinden van een modem met je PC, en met zo’n null-modemkabel kon je dan 2 computers rechtstreeks
verbinden zonder de modems ertussen.

\subsection{Protocol (Layer 2)}
Alle data bits bevinden zich tussen een start- en stopsignaal. De start bit is één 0-bit, de volgende
zeven of acht bits vormen een databyte. Eventueel kan er als controle nog een parity bit worden
toegevoegd. Het teken wordt telkens afgesloten door één of twee stop bits.

Aan de kant van de ontvanger wordt de data aan het start- en stopsignaal herkend.
Het aantal start-, data- en stopbits kan worden ingesteld en moet uiteraard langs beide kanten overeen komen.

\subsubsection{Baudrate}
= De snelheid = symbolen per seconde

\begin{itemize}
    \item Zender en ontvanger moeten dezelfde baudrate hebben
    \item Hogere rate $\rightarrow$ hogere eisen qua elektrische verbinding en timingnauwkeurigheid
    \item Lagere rate $\rightarrow$ minder snelle communicatie, maar grotere maximale kabellengte
    \item Positieve spanning: lage bit, negatieve spanning: hoge bit
\end{itemize}

\bold{Veelgebruikte baudrates:}
\begin{itemize}
    \item 9600 baud (=veelvoud van 300, vandaag standaard maar traag, wordt door vrijwel elke UART ondersteund)
    \item 19200, 57600, 115200 baud (= veelvouden van 9600)
    \item 64k baud en veelvouden
    \item 1M baud en veelvouden
\end{itemize}

\begin{figure}[H]
    \centering
    \includegraphics[width=0.5\textwidth]{serielecommunicatie.png}
    \caption{Timingdiagram seriele communicatie}
\end{figure}

\subsection{Flow control}
Mechanisme waarmee zender en ontvanger elkaar kunnen verwittigen wanneer ze klaar zijn voor datatransmissie. 
Dat kan zowel in hardware (dus met extra signaaldraden), of in software met speciale karakters.

\bold{Hardware:}
\begin{itemize}
    \item RTS/CTS: Request To Send/Clear To Send pins worden gebruikt om aan te geven dat het apparaat klaar is om data te sturen/ontvangen
    \item DSR/DTR: Data Set Ready/Data Terminal Ready-lijnen die eigenlijk aangeven dat het apparaat online en klaar is, worden soms misbruikt voor flow control
\end{itemize}

\bold{Software:}
\begin{itemize}
    \item XON/XOFF: Zijn speciale ASCII-controletekens die als data van ontvanger naar verzender worden gestuurd om transmissie te starten/stoppen
\end{itemize}

Flow control was van groot belang om een modem aan te sluiten maar wordt vandaag niet meer veel gebruikt.

\subsection{Verschillende opties seriele communicatie}
\begin{enumerate}
    \item USB kabel
    \item Gebruik van seriele pins op de GPIO connector
\end{enumerate}

\subsection{Byte serieel versturen}
Het serieel protocol heeft maar 1 draad om een byte (char) te versturen of ontvangen.
Daarom moeten we elke byte ontleden en bit per bit versturen. De meeste protocollen starten met de MSB (meest linkse bit).

\subsubsection{In python: PySerial}
= een library om makkelijk een seriele poort te openen:

\begin{lstlisting}[language=Python]
import serial
ser = serial.Serial('/dev/ttyS0') # open serial port
print(ser.name)                   # check which port was really used
ser.write(b'hello')               # write a string
ser.close()                       # close port
\end{lstlisting}

\begin{lstlisting}[language=Python]
from serial import Serial, PARITY_NONE

# simple echo server
with Serial('/dev/ttyS0', 115200, bytesize=8, parity=PARITY_NONE, stopbits=1) as port:
    while True:
        line = port.readline()
        port.write(line)
\end{lstlisting}

\subsection{Level shifter}
= een module die we kunnen gebruiken om componenten die op een verschillende 
spanning werken digitaal met elkaar te verbinden. 

\begin{itemize}
    \item De ene kant is low voltage (LV), de andere high voltage (HV)
\end{itemize}

\begin{figure}[H]
    \centering
    \includegraphics[width=0.2\textwidth]{levelshifter.png}
    \caption{Een levelshifter met 4 bits}
\end{figure}

\begin{figure}[H]
    \centering
    \includegraphics[width=0.4\textwidth]{levelshifter-arduino.png}
    \caption{Verbinding tussen Arduino en Pi. Let op de crossed cabling tussen Arduino en levelshifter}
\end{figure}

\section{Shiftregister}
Een shiftregister ontvangt een aantal bits in serie die parallel naar buiten gebracht worden. 
Op die manier kunnen we via één data pin van onze raspberry pi meerdere uitgangen creëren.

Een shiftregister is opgebouwd uit een aantal elektronische componenten (flip flops of latches) die in serie
geschakeld worden met elkaar en dezelfde klok delen. De uitgang van elke flip flop wordt gebruikt als ingang van de volgende. Zo’n schakelingen wordt ook een
cascade (waterval) schakeling genoemd. Op die manier kan je ook meerdere shift registers met elkaar
verbinden.

\subsection{Shiftregister 74HC595}
De 74HC595 is een high speed 8-bit `serieel in parallel out' shiftregister.
We kunnen dus 8 bits na elkaar serieel versturen naar het register, 
die bits worden vervolgens naar 8 parallelle uitgangen gebracht.

Dit shiftregister heeft ook een storage register dat de data bij kan houden. 
Voor het 7-segment display is dit handig omdat we de vorige waarde kunnen blijven tonen.

\subsection{Pinnummering}

\begin{figure}[H]
    \centering
    \includegraphics[width=0.3\textwidth]{shiftregister-pins.png}
    \caption{Pin configuratie 74HC595}
\end{figure}

Deze pins moeten we verbinden met een GPIO-pin van de RPi:
\begin{itemize}
    \item Master Reset (MR)
    \item De twee klokken (SHCP en STCP)
    \item De Output Enable (OE)
    \item De Serial Data (SD of $\text{D}_{\text{S}}$)
\end{itemize}

Q0 t.e.m. Q7 sluiten we aan ons component (hier: 7-segment display)

Door gebruik te maken van 5 GPIO pinnen beschikken we nu over 8 digitale uitgangen.
Door een shiftregister toe te voegen beschikken we over 16 uitgangen met onze 5 GPIO-pinnen.

\subsection{Protocol}

\begin{figure}[H]
    \centering
    \includegraphics[width=0.5\textwidth]{timing-shift.png}
    \caption{Timingdiagram shiftregister 74HC595}
\end{figure}


\subsection{7-segment display met shiftregister}
= display dat bestaat uit 8 segmenten (7 voor het cijfer + 1 voor het punt)

\subsubsection{Twee soorten}
\begin{enumerate}
    \item Common Cathode
    \item Common Anode
\end{enumerate}

\begin{figure}[H]
    \centering
    \includegraphics[width=0.45\textwidth]{7-display.png}
    \caption{7-segment display: twee soorten}
\end{figure}

\subsection{Shift input}
Sommige shiftregisters kunnen ook als input gebruikt worden $\rightarrow$ meerdere digitale inputs met slechts 5 pinnen.



\end{document}
